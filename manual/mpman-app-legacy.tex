\svnInfo $Id$
\section{Legacy Information}
\label{legacy}

\subsection{MetaPost Versus \MF}
\label{MPvsMF}

Since the \MF\index{metafont?\MF} and MetaPost languages have so much in
common, expert users of \MF\ will want to skip most of the explanations
in this document and concentrate on concepts that are unique to
MetaPost.  The comparisons in this appendix are intended to help experts
that are familiar with {\sl The\ \MF book} as well as other users that
want to benefit from Knuth's more detailed explanations \cite{kn:c}.

Since \MF\ is intended for making \TeX\ fonts, it has a number of
primitives for generating the {\tt tfm}\index{tfm file?{\tt tfm}
file}\index{files!tfm?{\tt tfm}} files that \TeX\ needs for character
dimensions, spacing information, ligatures\index{ligatures} and
kerning\index{kerning}.  MetaPost can also be used for generating fonts,
and it also has \MF's primitives for making {\tt tfm} files.  These are
listed in Table~\ref{tfmprim}.  Explanations can be found in the \MF\
documentation \cite{kn:c,kn:mf3}.

\begin{table}[htp]
$$\begin{tabular}{|l|l|}  \hline
commands&               {\tt charlist}, {\tt extensible},
                        {\tt fontdimen}, {\tt headerbyte}               \\
        &               {\tt kern}, {\tt ligtable}                      \\ \hline
ligtable operators&     \verb!::!, \verb!=:!, \verb!=:|!, \verb!=:|>!,
                        \verb!|=:!, \verb!|=:>!,                        \\
        &               \verb!|=:|!, \verb!|=:|>!, \verb!|=:|>>!,
                        \verb!||:!                                      \\ \hline
internal variables\index{internal variables}\index{variables!internal}&
                        {\tt boundarychar}, {\tt chardp},
                        {\tt charext}, {\tt charht},                    \\
        &               {\tt charic}, {\tt charwd},
                        {\tt designsize}, {\tt fontmaking}              \\ \hline
other operators&        {\tt charexists}                                \\ \hline
\end{tabular}
$$
\caption{MetaPost primitives for making {\tt tfm} files.}
\label{tfmprim}
\end{table}

Even though MetaPost has the primitives for generating fonts, many of
the font-making primitives and internal variables that are part of Plain
\MF\index{metafont?\MF} are not defined in Plain MetaPost\index{Plain
macros}.  Instead, there is a separate macro package called {\tt
mfplain}\index{mfplain?\texttt{mfplain}} that defines the macros
required to allow MetaPost to process Knuth's Computer Modern fonts as
shown in Table~\ref{mfponly} \cite{kn:e}.  To load these macros, put
``\verb|&mfplain|'' before the name of the input file.  This can be done
at the {\tt **} prompt after invoking the MetaPost interpreter with no
arguments, or on a command line that looks something like
this:\footnote{Command line syntax is system dependent.  Quotes are
needed on most Unix systems to protect special characters like
{\tt\&}.}
$$ \hbox{\verb|mpost '&mfplain' cmr10|} $$
The analog of a \MF\ command line like
$$ \hbox{\verb|mf '\mode=lowres; mag=1.2; input cmr10'|} $$
is
$$ \hbox{\verb|mpost '&mfplain \mode=lowres; mag=1.2; input cmr10'|} $$
The result is a set of PostScript\index{PostScript} files, one for each
character in the font.  Some editing would be required in order to
merge them into a downloadable Type~3 PostScript font~\cite{ad:red2}.

\begin{table}[htp]
$$
\renewcommand{\FancyVerbFormatLine}[1]{\hbox{#1}\strut}
\begin{tabular}{|l|} \hline
\multicolumn 1{|c|}
{Defined in the {\tt mfplain} package}          \\ \hline
\begin{verbatim}
beginchar                              font_identifier
blacker                                font_normal_shrink
capsule_def                            font_normal_space
change_width                           font_normal_stretch
define_blacker_pixels                  font_quad
define_corrected_pixels                font_size
define_good_x_pixels                   font_slant
define_good_y_pixels                   font_x_height
define_horizontal_corrected_pixels     italcorr
define_pixels                          labelfont
define_whole_blacker_pixels            makebox
define_whole_pixels                    makegrid
define_whole_vertical_blacker_pixels   maketicks
define_whole_vertical_pixels           mode_def
endchar                                mode_setup
extra_beginchar                        o_correction
extra_endchar                          proofrule
extra_setup                            proofrulethickness
font_coding_scheme                     rulepen
font_extra_space                       smode
\end{verbatim}
                                               \\ \hline
\multicolumn 1{|c|}
{Defined as no-ops in the {\tt mfplain} package}\\ \hline
\begin{verbatim}
cullit                                 proofoffset
currenttransform                       screenchars
gfcorners                              screenrule
grayfont                               screenstrokes
hround                                 showit
imagerules                             slantfont
lowres_fix                             titlefont
nodisplays                             unitpixel
notransforms                           vround
openit
\end{verbatim}
                                       \\ \hline
\end{tabular}
\renewcommand{\FancyVerbFormatLine}[1]{#1}
$$
\caption{Macros and internal variables defined only in the {\tt mfplain}
package.}
\label{mfponly}
\end{table}


\begin{table}
\centering
\begin{tabular}{|c|}\hline
MetaPost primitives not found in \MF\\\hline
\begin{minipage}[b]{.75\linewidth}\ttfamily
\begin{multicols}{3}
blackpart\\
bluepart\\
bounded\\
btex\\
clip\\
clipped\\
closefrom\\
cmykcolor\\
color\\
colormodel\\
cyanpart\\
dashed\\
dashpart\\
defaultcolormodel\\
etex\\
filenametemplate\\
filled\\
fontmapfile\\
fontmapline\\
fontpart\\
fontsize\\
for within\\
glyph of\\
greenpart\\
greypart\\
hour\\
infont\\
jobname\\
linecap\\
linejoin\\
llcorner\\
lrcorner\\
magentapart\\
minute\\
miterlimit\\
mpprocset\\
mpxbreak\\
numberprecision\\
numbersystem\\
outputfilename\\
outputformat\\
outputformatoptions\\
outputtemplate\\
pathpart\\
penpart\\
prologues\\
readfrom\\
redpart\\
restoreclipcolor\\
rgbcolor\\
setbounds\\
stroked\\
textpart\\
textual\\
tracinglostchars\\
troffmode\\
truecorners\\
ulcorner\\
urcorner\\
verbatimtex\\
withcmykcolor\\
withcolor\\
withgreyscale\\
withoutcolor\\
withpostscript\\
withprescript\\
withrgbcolor\\
write to\\
yellowpart
\end{multicols}
\end{minipage}\\\hline

Variables and Macros defined only in Plain MetaPost\\\hline
\begin{minipage}[b]{.75\linewidth}\ttfamily
\begin{multicols}{3}
ahangle\\
ahlength\\
background\\
bbox\\
bboxmargin\\
beginfig\\
beveled\\
black\\
blue\\
buildcycle\\
butt\\
center\\
colorpart\\
cutafter\\
cutbefore\\
cuttings\\
dashpattern\\
defaultfont\\
defaultpen\\
defaultscale\\
dotlabel\\
dotlabels\\
drawarrow\\
drawdblarrow\\
drawoptions\\
endfig\\
EOF\\
evenly\\
extra\_beginfig\\
extra\_endfig\\
green\\
image\\
label\\
labeloffset\\
mitered\\
red\\
rounded\\
squared\\
thelabel\\
white
\end{multicols}
\end{minipage}\\\hline
\end{tabular}
\caption{Macros and internal variables defined in MetaPost but not \MF.}
\label{mponly}
\end{table}

Another limitation of the {\tt mfplain} package is that certain internal
variables from Plain \MF\index{metafont?\MF} cannot be given reasonable
MetaPost definitions.  These include {\tt displaying}, {\tt
currentwindow}, \verb|screen_rows|, and \verb|screen_cols| which depend
on \MF's ability to display images on the computer screen.  In addition,
\verb|pixels_per_inch| is irrelevant since MetaPost uses fixed units of
PostScript\index{PostScript!point}\index{point!PostScript} points.

The reason why some macros and internal variables\index{internal
variables}\index{variables!internal} are not meaningful in MetaPost is
that \MF\ primitive commands {\tt cull}, {\tt display}, {\tt
openwindow}, {\tt numspecial} and {\tt totalweight} are not implemented
in MetaPost.  Also not implemented are a number of internal variables as
well as the \tdescr{drawing option} {\tt withweight}.  Here is a
complete listing of the internal variables whose primitive meanings in
\MF\ do not make sense in MetaPost:
$$\begin{verbatim}
autorounding   fillin         smoothing      turningcheck
chardx         granularity    tracingedges   xoffset
chardy         proofing       tracingpens    yoffset
\end{verbatim}
$$

There is also one \MF\ primitive that has a slightly different meaning in
MetaPost.  Both languages allow statements of the
form\index{special?\texttt{special}}\label{Dspecl}
$$ {\tt special}\, \descr{string expression} \hbox{\tt;} $$
but \MF\ copies the string into its ``generic font'' output file, while
MetaPost interprets the string as a sequence of PostScript\index{PostScript}
commands that are to be placed at the beginning of the next output file.

In this regard, it is worth mentioning that rules in \TeX\ material
included via {\tt btex..etex} in MetaPost are rounded to the correct
number of pixels according to PostScript\index{PostScript!conversion rules}
conversion rules~\cite{ad:red2}.  In \MF, rules are not generated
directly, but simply included in specials and interpreted later by
other programs, such as {\tt gftodvi}\index{gftodvi?\ttt{gftodvi}},
so there is no special conversion.

All the other differences between \MF\ and MetaPost are features found
only in MetaPost.  These are listed in Table~\ref{mponly}.  The only
commands listed in this table that the preceding sections do not discuss
are
\verb|extra_beginfig|\index{extra_beginfig?\texttt{extra\_beginfig}}\label{Dxbfig},
\verb|extra_endfig|\index{extra_endfig?\texttt{extra\_endfig}}\label{Dxefig},
and {\tt mpxbreak}.  The first two are strings that contain extra
commands to be processed by {\tt
beginfig}\index{beginfig?\texttt{beginfig}} and {\tt
endfig}\index{endfig?\texttt{endfig}} just as \verb|extra_beginchar| and
\verb|extra_endchar| are processed by {\tt beginchar} and {\tt endchar}.
(The file {\tt boxes.mp}\index{boxes.mp?\texttt{boxes.mp}} uses these
features).

The other new feature listed in Table~\ref{mponly} not listed in the
index is {\tt mpxbreak}\index{mpxbreak?\texttt{mpxbreak}}.  This is used
to separate blocks of translated \TeX\index{TeX?\TeX} or
troff\index{troff} commands in {\tt mpx}\index{mpx file?{\tt mpx}
  file}\index{files!mpx?{\tt mpx}} files.  It should be of no concern to
users since {\tt mpx} files are generated automatically.

\subsection{File Name Templates}
\label{Dfilenametemplate}

The output file naming template mechanism introduced in MetaPost
version~1.000 originally used a primitive called
\ttindex{filenametemplate}, as opposed to the internal string variable
\ttt{outputtemplate}\index{outputtemplate?\texttt{outputtemplate}}
described in section~\ref{Doutputtemplate}.  This primitive took a
string argument with the same syntax as \ttt{outputtemplate}, except
that it didn't know about the \verb|%{...}| escape sequence for
evaluating internal variables, e.g.,:

$$\begin{verbatim}
filenametemplate "%j-%c.mps";
\end{verbatim}
$$

The \verb|filenametemplate| primitive has been deprecated since the
introduction of \verb|outputtemplate| (version~1.200), but is still
supported.  If you happen to need writing future-proof source files,
that at the same time are backwards compatible to MetaPost versions
between 1.000 and~1.200, this output filename template declaration might
help:

$$\begin{verbatim}
if scantokens(mpversion) < 1.200:
  filenametemplate
else:
  outputtemplate :=
fi
"%j-%c.mps";
\end{verbatim}
$$


%%% Local Variables: 
%%% mode: latex
%%% TeX-PDF-mode: t
%%% TeX-master: "mpman"
%%% End: 
