\svnInfo $Id$
\section{Reference Manual}
\label{refman}

\begingroup
\renewcommand\topfraction{1.0}% Allow *lots* of figures and tables.
\renewcommand\textfraction{0.0}
\setcounter{topnumber}{10}
\setcounter{totalnumber}{10}

\subsection{The MetaPost Language}
\label{refman:lang}

Tables \ref{ivartab}--\ref{pseudotab} summarize the built-in features of
Plain MetaPost.  Features from the Plain\index{Plain macros} macro
package are marked by \pl\ symbols.  The distinction between primitives
and plain macros can be ignored by the casual user.

The tables in this appendix give the name of each feature, the page
number where it is explained, and a short description.  A few features
are not explained elsewhere and have no page number listed.  These
features exist primarily for compatibility with \MF\index{metafont?\MF}
and are intended to be self-explanatory.  Certain other features from
\MF\ are omitted entirely because they are of limited interest to the
MetaPost users and/or would require long explanations.  All of these are
documented in {\sl The \MF book} \cite{kn:c} as explained in
Appendix~\ref{MPvsMF}.

Tables~\ref{ivartab} and~\ref{isvartab} list internal variables that
take on numeric and string values.  Table~\ref{pvartab} lists predefined
variables of other types.  Table~\ref{consttab} lists predefined
constants.  Some of these are implemented as variables whose values are
intended to be left unchanged.

Table~\ref{optab} summarizes MetaPost operators and lists
the possible argument and result types for each one.  A ``--'' entry for
the left argument indicates a unary operator; ``--'' entries for both
arguments indicate a nullary operator.  Operators that take suffix
parameters are not listed in this table because they are treated as
``function-like macros''.

The last two tables are Table~\ref{cmdtab} for commands and
Table~\ref{pseudotab} macros that behave like functions or procedures.
Such macros take parenthesized argument lists and/or suffix parameters,
returning either a value whose type is listed in the table, or nothing.
The latter case is for macros that behave like procedures.  Their return
values are listed as ``--''.

The figures in this appendix present the syntax of the MetaPost language
starting with expressions in Figures \ref{syexpr1}--\ref{sypseudo}.
Although the productions sometimes specify types for expressions, primaries,
secondaries, and tertiaries, no attempt is made to give separate syntaxes
for \tdescr{numeric expression}, \tdescr{pair expression}, etc.
The simplicity of the productions in Figure~\ref{sytypexpr} is due to this
lack of type information.  Type information
can be found in Tables \ref{ivartab}--\ref{pseudotab}.

Figures \ref{syprog}, \ref{sycmds1} and~\ref{sycmds2} give the syntax
for MetaPost programs, including statements and commands.  They do not
mention loops\index{loops} and {\tt if}\index{if?\texttt{if}} tests
because these constructions do not behave like statements.  The syntax
given in Figures \ref{syexpr1}--\ref{sycondloop} applies to the result
of expanding all conditionals and loops.  Conditionals and loops do have
a syntax, but they deal with almost arbitrary sequences of tokens.
Figure~\ref{sycondloop} specifies conditionals in terms of
\tdescr{balanced tokens} and loops in terms of \tdescr{loop text}, where
\tdescr{balanced tokens} is any sequence of tokens balanced with respect
to {\tt if} and {\tt fi}, and \tdescr{loop text} is a sequence of tokens
balanced with respect to {\tt for}, {\tt forsuffixes}, {\tt forever},
and {\tt endfor}.

\begin{table}[htp]
\caption{Internal variables with numeric values}
$$\begin{tabular}{|l|r|l|}
\hline
\multicolumn1{|c}{Name}&  \multicolumn1{|c}{Page}&  \multicolumn1{|c|}{Explanation}\\
\hline
\hline
\pl\tt  ahangle&        \pageref{Dahangle}&
        Angle for arrowheads in degrees (default: 45)\\\hline
\pl\tt  ahlength&       \pageref{Dahlength}&
        Size of arrowheads (default: 4{\tt bp})\\\hline
\pl\tt  bboxmargin&     \pageref{Dbbmargin}&
        Extra space allowed by {\tt bbox} (default 2{\tt bp})\\\hline
\tt     charcode&       \pageref{Dcharcode}&
        The number of the current figure\\\hline
\tt     day&            \pageref{Dday}&
        The current day of the month\\\hline
\tt     defaultcolormodel&    \pageref{Ddefaultcolormodel}&
        The initial color model  (default: 5, rgb)\\\hline
\pl\tt  defaultpen&     \pageref{Ddefaultpen}&
        Numeric index used by {\tt pickup} to select default pen\\\hline
\pl\tt  defaultscale&   \pageref{Ddfscale}&
        Font scale factor for label strings (default 1)\\\hline
\pl\tt  dotlabeldiam&   \pageref{Ddotlabdiam}&
        Diameter of the dot drawn by \verb|dotlabel| (default 3{\tt bp})\\\hline
\tt     hour&           \pageref{Dhour}&
        The hour of the day this job started\\\hline
\tt     hppp&           \pageref{Dhppp}&
        The number of horizontal points per pixel for bitmap output\\\hline
\pl\tt  labeloffset&    \pageref{Dlaboff}&
        Offset distance for labels (default 3{\tt bp})\\\hline
\tt     linecap&        \pageref{Dlinecap}&
        0 for butt, 1 for round, 2 for square\\\hline
\tt     linejoin&       \pageref{Dlinejoin}&
        0 for mitered, 1 for round, 2 for beveled\\\hline
\tt     minute&         \pageref{Dminute}&
        The minute of the hour this job started\\\hline
\tt     miterlimit&     \pageref{Dmiterlim}&
        Controls miter length as in PostScript\\\hline
\tt     month&          \pageref{Dmonth}&
        The current month (e.g, 3 $\equiv$ March)\\\hline
\tt     mpprocset&      \pageref{Dmpprocset}&
        Create a PostScript dictionary of command abbreviations\\\hline
\tt     numberprecision&\pageref{Dnumberprecision}&
        Arithmetic precision in \texttt{binary} and \texttt{decimal} mode\\\hline
\tt     pausing&        --&
        ${}>0$ to display lines on the terminal before they are read\\\hline
\tt     prologues&      \pageref{Dprologues}&
        ${}>0$ to output conforming PostScript using built-in fonts\\\hline
\tt     restoreclipcolor&    --&
        Restore the graphics state after clip operations (default: 1)\\\hline
\tt     showstopping&   --&
        ${}>0$ to stop after each {\tt show} command\\\hline
\tt     time&           \pageref{Dtime}&
        The number of minutes past midnight when this job started\\\hline
\tt     tracingcapsules&        \pageref{Dtcapsules}&
        ${}>0$ to show capsules too\\\hline
\tt     tracingchoices& \pageref{Dtchoices}&
        ${}>0$ to show the control points chosen for paths\\\hline
\tt     tracingcommands&        \pageref{Dtcommands}&
        ${}>0$ to show commands and operations as they are performed\\\hline
\tt     tracingequations&       \pageref{Dtequations}&
        ${}>0$ to show each variable when it becomes known\\\hline
\tt     tracinglostchars&       \pageref{Dtlostchars}&
        ${}>0$ to show characters that aren't {\tt infont}\\\hline
\tt     tracingmacros&  \pageref{Dtmacros}&
        ${}>0$ to show macros before they are expanded\\\hline
\tt     tracingonline&  \pageref{Dtonline}&
        ${}>0$ to show long diagnostics on the terminal\\\hline
\tt     tracingoutput&  \pageref{Dtoutput}&
        ${}>0$ to show digitized edges as they are output\\\hline
\tt     tracingrestores&        \pageref{Dtrestores}&
        ${}>0$ to show when a variable or internal is restored\\\hline
\tt     tracingspecs&   \pageref{Dtspecs}&
        ${}>0$ to show path subdivision when using a polygonal a pen\\\hline
\tt     tracingstats&   \pageref{Dtstats}&
        ${}>0$ to show memory usage at end of job\\\hline
\tt     tracingtitles&  --&
        ${}>0$ to show titles online when they appear\\\hline
\tt     troffmode&    \pageref{Dtroffmode}&
        Set to~1 if a {\tt -troff} or {\tt -T} option was given\\\hline
\tt     truecorners&    \pageref{Dtruecorn}&
        ${}>0$ to make {\tt llcorner} etc. ignore {\tt setbounds}\\\hline
\tt     vppp&           \pageref{Dvppp}&
        The number of vertical points per pixel for bitmap output\\\hline
\tt     warningcheck&   \pageref{Dwarncheck}&
        Controls error message when variable value is large\\\hline
\tt     year&           \pageref{Dyear}&
        The current year (e.g., 1992)\\\hline
\end{tabular}
$$
\label{ivartab}%
\index{pausing?\texttt{pausing}}\index{showstopping?\texttt{showstopping}}%
\index{tracingtitles?\texttt{tracingtitles}}
\end{table}


\begin{table}[htp]
\caption{Internal string variables}
$$\begin{tabular}{|l|r|l|}
\hline
\multicolumn1{|c}{Name}&  \multicolumn1{|c}{Page}&  \multicolumn1{|c|}{Explanation}\\
\hline
\hline
\tt     jobname&    --&
        The name of this job\\\hline
\tt     numbersystem&    \pageref{Dnumbersystem}&
        Arithmetic mode as given on the command-line\\\hline
\tt     outputfilename&    \pageref{Doutputfilename}&
        Name of the output file last written\\\hline
\tt     outputformat&    \pageref{Doutputformat}&
        Output backend to be used (default: ``eps'')\\\hline
\tt     outputformatoptions&    \pageref{Doutputformatoptions}&
        Configure PNG output backend (default: ``'')\\\hline
\tt     outputtemplate&    \pageref{Doutputtemplate}&
        Output filename template (default: ``\%j.\%c'')\\\hline
\end{tabular}
$$
\label{isvartab}
\index{jobname?\texttt{jobname}}
\end{table}


\begin{table}[htp]
\caption{Other Predefined Variables}
$$\begin{tabular}{|l|l|r|l|}
\hline
\multicolumn1{|c}{Name}&  \multicolumn1{|c}{Type}&  \multicolumn1{|c}{Page}&
        \multicolumn1{|c|}{Explanation}\\
\hline
\hline
\pl\tt  background&     color&  \pageref{Dbground}&
        Color for {\tt unfill} and {\tt undraw} (usually white)\\\hline
\pl\tt  currentpen&     pen&    \pageref{Dcurpen}&
        Last pen picked up (for use by the {\tt draw} command)\\\hline
\pl\tt  currentpicture& picture&        \pageref{Dcurpic}&
        Accumulate results of {\tt draw} and {\tt fill} commands\\\hline
\pl\tt  cuttings&       path&   \pageref{Dcuttings}&
        Subpath cut off by last {\tt cutbefore} or {\tt cutafter}\\\hline
\pl\tt  defaultfont&    string& \pageref{Ddffont}&
        Font used by label commands for typesetting strings\\\hline
\pl\tt  extra\_beginfig&        string& \pageref{Dxbfig}&
        Commands for {\tt beginfig} to scan\\\hline
\pl\tt  extra\_endfig&  string& \pageref{Dxefig}&
        Commands for {\tt endfig} to scan\\\hline
\end{tabular}
$$
\label{pvartab}
\end{table}


\begin{table}[htp]
\caption{Predefined Constants}
$$\begin{tabular}{|l|l|r|l|}
\hline
\multicolumn1{|c}{Name}&  \multicolumn1{|c}{Type}&  \multicolumn1{|c}{Page}&
        \multicolumn1{|c|}{Explanation}\\
\hline
\hline
\pl\tt  beveled&        numeric&        \pageref{Dbvled}&
        {\tt linejoin} value for beveled joins [2]\\\hline
\pl\tt  black&  color&  \pageref{Dblack}&
        Equivalent to {\tt (0,0,0)}\\\hline
\pl\tt  blue&   color&  \pageref{Dblue}&
        Equivalent to {\tt (0,0,1)}\\\hline
\pl\tt  bp&     numeric&        \pageref{Dbp}&
        One PostScript point in {\tt bp} units [1]\\\hline
\pl\tt  butt&   numeric&        \pageref{Dbutt}&
        {\tt linecap} value for butt end caps [0]\\\hline
\pl\tt  cc&     numeric&        \pageref{Dcc}&
        One cicero in {\tt bp} units [12.79213]\\\hline
\pl\tt  cm&     numeric&        \pageref{Dcm}&
        One centimeter in {\tt bp} units [28.34645]\\\hline
\pl\tt  dd&     numeric&        \pageref{Ddd}&
        One didot point in {\tt bp} units [1.06601]\\\hline
\pl\tt  ditto&  string& --&
        The {\tt "} character as a string of length 1\\\hline
\pl\tt  down&   pair&   \pageref{Ddown}&
        Downward direction vector $(0,-1)$\\\hline
\pl\tt  epsilon&        numeric&        \pageref{Depsilon}&
        Smallest positive MetaPost number [$1\over65536$]\\\hline
\pl\tt  evenly& picture&        \pageref{Devenly}&
        Dash pattern for equal length dashes\\\hline
\pl\tt  EOF& string&        \pageref{Deof}&
        Single null character\\\hline
\tt     false&  boolean&        \pageref{Dfalse}&
        The boolean value {\it false\/}\\\hline
\pl\tt  fullcircle&     path&   \pageref{Dfcirc}&
        Circle of diameter 1 centered on $(0,0)$\\\hline
\pl\tt  green&  color&  \pageref{Dgreen}&
        Equivalent to {\tt (0,1,0)}\\\hline
\pl\tt  halfcircle&     path&   \pageref{Dhcirc}&
        Upper half of a circle of diameter 1\\\hline
\pl\tt  identity&       transform&      \pageref{Dident}&
        Identity transformation\\\hline
\pl\tt  in&     numeric&        \pageref{Din}&
        One inch in {\tt bp} units [72]\\\hline
\pl\tt  infinity&       numeric&        \pageref{Dinf}&
        Large positive value [4095.99998]\\\hline
\pl\tt  left&   pair&   \pageref{Dleft}&
        Leftward direction $(-1,0)$\\\hline
\pl\tt  mitered&        numeric&        \pageref{Dmitred}&
        {\tt linejoin} value for mitered joins [0]\\\hline
\pl\tt  mm&     numeric&        \pageref{Dmm}&
        One millimeter in {\tt bp} units [2.83464]\\\hline
\tt     mpversion&  string&   \pageref{Dmpversion}&
        MetaPost version number\\\hline
\tt     nullpen&    pen&        \pageref{Dnlpen}&
        Empty pen\\\hline
\tt     nullpicture&    picture&        \pageref{Dnlpic}&
        Empty picture\\\hline
\pl\tt  origin& pair&   \pageref{Dorigin}&
        The pair $(0,0)$\\\hline
\pl\tt  pc&     numeric&        \pageref{Dpc}&
        One pica in {\tt bp} units [11.95517]\\\hline
\tt     pencircle&      pen&    \pageref{Dpncirc}&
        Circular pen of diameter 1\\\hline
\pl\tt  pensquare&      pen&    \pageref{Dpnsqr}&
        Square pen of height 1 and width 1\\\hline
\pl\tt  pt&     numeric&        \pageref{Dpt}&
        One printer's point in {\tt bp} units [0.99626]\\\hline
\pl\tt  quartercircle&  path&   \pageref{Dqcirc}&
        First quadrant of a circle of diameter 1\\\hline
\pl\tt  red&    color&  \pageref{Dred}&
        Equivalent to {\tt (1,0,0)}\\\hline
\pl\tt  right&  pair&   \pageref{Dright}&
        Rightward direction $(1,0)$\\\hline
\pl\tt  rounded&        numeric&        \pageref{Drnded}&
        {\tt linecap} and {\tt linejoin} value for round joins\\
\tt     &       &       &
        and end caps [1]\\\hline
\pl\tt  squared&        numeric&        \pageref{Dsqred}&
        {\tt linecap} value for square end caps [2]\\\hline
\tt     true&   boolean&        \pageref{Dtrue}&
        The boolean value {\tt true}\\\hline
\pl\tt  unitsquare&     path&   \pageref{Dunitsqr}&
        The path \verb|(0,0)--(1,0)--(1,1)--(0,1)--cycle|\\\hline
\pl\tt  up&     pair&   \pageref{Dup}&
        Upward direction $(0,1)$\\\hline
\pl\tt  white&  color&  \pageref{Dwhite}&
        Equivalent to {\tt (1,1,1)}\\\hline
\pl\tt  withdots&       picture&        \pageref{Dwdots}&
        Dash pattern that produces dotted lines\\\hline
\end{tabular}
$$
\label{consttab}%
\index{ditto?\texttt{ditto}}%
\end{table}


\clearpage
\LTXtable{\textwidth}{mpman-app-optab}

\begin{table}[htp]
\caption{Commands}
$$\begin{tabular}{|l|r|l|}
\hline
\multicolumn1{|c}{Name}&  \multicolumn1{|c}{Page}&  \multicolumn1{|c|}{Explanation}\\
\hline
\hline
\tt \verb|addto|&       \pageref{sydraw}&
        Low-level command for drawing and filling\\\hline
\tt \verb|clip|&        \pageref{Dclip}&
        Applies a clipping path to a picture\\\hline
\tt \verb|closefrom|&  \pageref{Dclosefrom}&
        Close a file opened by {\tt readfrom}\\\hline
\pl\tt \verb|cutdraw|&  \pageref{Dctdraw}&
        Draw with butt end caps\\\hline
\tt \verb|dashed|&     \pageref{Ddashed}&
        Apply dash pattern to drawing command\\\hline
\pl\tt \verb|draw|&     \pageref{curves}&
        Draw a line or a picture\\\hline
\pl\tt \verb|drawarrow|&        \pageref{Ddrwarr}&
        Draw a line with an arrowhead at the end\\\hline
\pl\tt \verb|drawdblarrow|&     \pageref{Ddrwdar}&
        Draw a line with arrowheads at both ends\\\hline
\tt \verb|errhelp|& \pageref{Derrhelp}&
        Declare help message for interactive mode\\\hline
\tt \verb|errmessage|& \pageref{Derrmessage}&
        Show error message on the terminal and enter interactive
        mode\\\hline
\tt \verb|filenametemplate|& \pageref{Dfilenametemplate}&
        Set output file name pattern (deprecated, see
        \verb|outputtemplate|)\\\hline
\pl\tt \verb|fill|&     \pageref{Dfill}&
        Fill inside a cyclic path\\\hline
\pl\tt \verb|filldraw|& \pageref{Dfildrw}&
        Draw a cyclic path and fill inside it\\\hline
\tt \verb|fontmapfile|&     \pageref{Dfontmapfile}&
        Read font map entries from file\\\hline
\tt \verb|fontmapline|&     \pageref{Dfontmapline}&
        Declare a font map entry\\\hline
\tt \verb|interim|&     \pageref{Dinterm}&
        Make a local change to an internal variable\\\hline
\tt \verb|let|& --&
        Assign one symbolic token the meaning of another\\\hline
\pl\tt \verb|loggingall|&       \pageref{Dlogall}&
        Turn on all tracing (log file only)\\\hline
\tt \verb|message|& \pageref{Dmessage}&
        Show message string on the terminal\\\hline
\tt \verb|newinternal|& \pageref{Dnewint}&
        Declare new internal variables\\\hline
\pl\tt \verb|pickup|&   \pageref{Dpickup}&
        Specify new pen for line drawing\\\hline
\tt \verb|save|&        \pageref{Dsave}&
        Make variables local\\\hline
\tt \verb|setbounds|&   \pageref{Dsetbnd}&
        Make a picture lie about its bounding box\\\hline
\tt \verb|shipout|&     \pageref{Dship}&
        Low-level command to output a figure\\\hline
\tt \verb|show|&        \pageref{Dshow}&
        Print out expressions symbolically\\\hline
\tt \verb|showdependencies|&    \pageref{Dshdep}&
        Print out all unsolved equations\\\hline
\tt \verb|showtoken|&   \pageref{Dshtok}&
        Print an explanation of what a token is\\\hline
\tt \verb|showvariable|&        \pageref{Dshvar}&
        Print variables symbolically\\\hline
\tt \verb|special|&     \pageref{Dspecl}&
        Print a string directly in the PostScript output file\\\hline
\pl\tt \verb|tracingall|&       \pageref{Dtall}&
        Turn on all tracing\\\hline
\pl\tt \verb|tracingnone|&      \pageref{Dtnone}&
        Turn off all tracing\\\hline
\pl\tt \verb|undraw|&   \pageref{Dundraw}&
        Erase a line or a picture\\\hline
\pl\tt \verb|unfill|&   \pageref{Dunfill}&
        Erase inside a cyclic path\\\hline
\pl\tt \verb|unfilldraw|&       \pageref{Dunfdrw}&
        Erase a cyclic path and its inside\\\hline
\tt \verb|withcmykcolor|&    \pageref{Dwithcmykcolor}&
        Apply CMYK color to drawing command\\\hline
\tt \verb|withcolor|&        \pageref{Dwithcolor}&
        Apply generic color specification to drawing command\\\hline
\tt \verb|withgreyscale|&    \pageref{Dwithgreyscale}&
        Apply greyscale color to drawing command\\\hline
\tt \verb|withoutcolor|&     \pageref{Dwithoutcolor}&
        Don't apply any color specification to drawing command\\\hline
\tt \verb|withpen|&          \pageref{sec.pens}&
        Apply pen to drawing operation\\\hline
\tt \verb|withpostscript|&   \pageref{Dwithpost}&
        End raw PostScript code\\\hline
\tt \verb|withprescript|&    \pageref{Dwithpre}&
        Begin raw PostScript code\\\hline
\tt \verb|withrgbcolor|&     \pageref{Dwithrgbcolor}&
        Apply RGB color to drawing command\\\hline
\tt \verb|write to|&         \pageref{Dwrite}&
        Write string to file\\\hline
\end{tabular}
$$
\index{let?\texttt{let}}%
\label{cmdtab}
\end{table}


\begin{table}[htp]
\caption{Function-Like Macros}
$$\begin{tabular}{|l|l|l|r|l|}
\hline
\multicolumn1{|c}{Name}&  \multicolumn1{|c}{Arguments}&
        \multicolumn1{|c}{Result}&  \multicolumn1{|c}{Page}&
        \multicolumn1{|c|}{Explanation}\\
\hline
\hline
\pl\tt \verb|buildcycle|&       list of paths&  path&   \pageref{buildcy}&
        Build a cyclic path\\\hline
\pl\tt \verb|dashpattern|&      on/off distances&       picture&        \pageref{Ddshpat}&
        Create a pattern for dashed lines\\\hline
\pl\tt \verb|decr|&     numeric variable&       numeric&        \pageref{Dincr}&
        Decrement and return new value\\\hline
\pl\tt \verb|dotlabel|& suffix, picture, pair&  --&     \pageref{Ddotlab}&
        Mark point and draw picture nearby\\\hline
\pl\tt \verb|dotlabel|& suffix, string, pair&   --&     \pageref{Ddotlab}&
        Mark point and place text nearby\\\hline
\pl\tt \verb|dotlabels|&        suffix, point numbers&  --&     \pageref{Ddotlbs}&
        Mark {\tt z} points with their numbers\\\hline
\pl\tt \verb|drawdot|&          pair&       --&     \pageref{Ddrawdot}&
        Draw a dot at the given point\\\hline
\pl\tt \verb|drawoptions|&      drawing options&        --&     \pageref{Ddropts}&
        Set options for drawing commands\\\hline
\pl\tt \verb|image|&              string&       picture&      \pageref{Dimage}&
        Return picture from text\\\hline
\pl\tt \verb|incr|&     numeric variable&       numeric&       \pageref{Dincr}&
        Increment and return new value\\\hline
\pl\tt \verb|label|&    suffix, picture, pair&  --&     \pageref{Dlabel}&
        Draw picture near given point\\\hline
\pl\tt \verb|label|&    suffix, string, pair&   --&     \pageref{Dlabel}&
        Place text near given point\\\hline
\pl\tt \verb|labels|&   suffix, point numbers&  --&     \pageref{Dlabels}&
        Draw {\tt z} point numbers; no dots\\\hline
\pl\tt \verb|max|&      list of numerics&       numeric&        --&
        Find the maximum\\\hline
\pl\tt \verb|max|&      list of strings&        string& --&
        Find the lexicographically last string\\\hline
\pl\tt \verb|min|&      list of numerics&       numeric&        --&
        Find the minimum\\\hline
\pl\tt \verb|min|&      list of strings&        string& --&
        Find the lexicographically first string\\\hline
\pl\tt \verb|thelabel|& suffix, picture, pair&  picture&        \pageref{Dthelab}&
        Picture shifted as if to label a point\\\hline
\pl\tt \verb|thelabel|& suffix, string, pair&   picture&        \pageref{Dthelab}&
        Text positioned as if to label a point\\\hline
\pl\tt \verb|z|&        suffix& pair&   \pageref{Dzconv}&
        The pair $({\tt x}\descr{suffix},{\tt y}\descr{suffix})$\\\hline
\end{tabular}
$$
\index{min?\texttt{min}}\index{max?\texttt{max}}%
\label{pseudotab}
\end{table}

\clearpage

\begin{figure}[htp]
\begin{ctabbing}
$\tt \descr{atom} \rightarrow \descr{variable} \;|\; \descr{argument}$\\
$\tt \qquad \;|\; \descr{number or fraction}$\\
$\tt \qquad \;|\; \descr{internal variable}$\\
$\tt \qquad \;|\; \hbox{\tt (}\descr{expression}\hbox{\tt )}$\\
$\tt \qquad \;|\; begingroup \descr{statement list} \descr{expression} endgroup$\\
$\tt \qquad \;|\; \descr{nullary op}$\\
$\tt \qquad \;|\; btex \descr{typesetting commands} etex$\\
$\tt \qquad \;|\; \descr{pseudo function}$\\
$\tt \descr{primary} \rightarrow \descr{atom}$\\
$\tt \qquad \;|\; \hbox{\tt (}\descr{numeric expression}\hbox{\tt ,} \descr{numeric expression}\hbox{\tt )}$\\
$\tt \qquad \;|\; \hbox{\tt (}\descr{numeric expression}\hbox{\tt ,} \descr{numeric expression}\hbox{\tt ,} \descr{numeric expression}\hbox{\tt )}$\\
$\tt \qquad \;|\; \descr{of operator} \descr{expression} of \descr{primary}$\\
$\tt \qquad \;|\; \descr{unary op} \descr{primary}$\\
$\tt \qquad \;|\; str \descr{suffix}$\\
$\tt \qquad \;|\; z \descr{suffix}$\\
$\tt \qquad \;|\; \descr{numeric atom}\hbox{\tt [}\descr{expression}\hbox{\tt ,}\descr{expression}\hbox{\tt ]}$\\
$\tt \qquad \;|\; \descr{scalar multiplication op} \descr{primary}$\\
$\tt \descr{secondary} \rightarrow \descr{primary}$\\
$\tt \qquad \;|\; \descr{secondary} \descr{primary binop} \descr{primary}$\\
$\tt \qquad \;|\; \descr{secondary} \descr{transformer}$\\
$\tt \descr{tertiary} \rightarrow \descr{secondary}$\\
$\tt \qquad \;|\; \descr{tertiary} \descr{secondary binop} \descr{secondary}$\\
$\tt \descr{subexpression} \rightarrow \descr{tertiary}$\\
$\tt \qquad \;|\; \descr{path expression} \descr{path join} \descr{path knot}$\\
$\tt \descr{expression} \rightarrow \descr{subexpression}$\\
$\tt \qquad \;|\; \descr{expression} \descr{tertiary binop} \descr{tertiary}$\\
$\tt \qquad \;|\; \descr{path subexpression} \descr{direction specifier}$\\
$\tt \qquad \;|\; \descr{path subexpression} \descr{path join} cycle$\\
$\tt $\\
$\tt \descr{path knot} \rightarrow \descr{tertiary}$\\
$\tt \descr{path join} \rightarrow \verb|--|$\\
$\tt \qquad \;|\; \descr{direction specifier} \descr{basic path join} \descr{direction specifier}$\\
$\tt \descr{direction specifier} \rightarrow \descr{empty}$\\
$\tt \qquad \;|\; \verb|{|curl \descr{numeric expression}\verb|}|$\\
$\tt \qquad \;|\; \verb|{|\descr{pair expression}\verb|}|$\\
$\tt \qquad \;|\; \verb|{|\descr{numeric expression}\hbox{\tt ,}\descr{numeric expression}\verb|}|$\\
$\tt \descr{basic path join} \rightarrow \hbox{\tt ..} \;|\; \hbox{\tt ...} \;|\; \hbox{\tt ..}\descr{tension}\hbox{\tt ..} \;|\; \hbox{\tt ..}\descr{controls}\hbox{\tt ..}$\\
$\tt \descr{tension} \rightarrow tension \descr{numeric primary}$\\
$\tt \qquad \;|\; tension \descr{numeric primary} and \descr{numeric primary}$\\
$\tt \descr{controls} \rightarrow controls \descr{pair primary}$\\
$\tt \qquad \;|\; controls \descr{pair primary} and \descr{pair primary}$\\
$\tt $\\
$\tt \descr{argument} \rightarrow \descr{symbolic token}$\\
$\tt \descr{number or fraction} \rightarrow \descr{number}\hbox{\tt /}\descr{number}$\\
$\tt \qquad \;|\; \descr{number not followed by `\hbox{\tt /}\tdescr{number}'}$\\
$\tt \descr{scalar multiplication op} \rightarrow + \;|\; -$\\
$\tt \qquad \;|\; \descr{`\tdescr{number or fraction}' not followed by `\tdescr{add op}\tdescr{number}'}$
\end{ctabbing}
\caption{Part 1 of the syntax for expressions}
\index{expression?\tdescr{expression}}\index{nullary op?\tdescr{nullary op}}\index{of operator?\tdescr{of operator}}%
\index{path knot?\tdescr{path knot}}\index{primary?\tdescr{primary}}\index{primary binop?\tdescr{primary binop}}%
\index{secondary?\tdescr{secondary}}\index{secondary binop?\tdescr{secondary binop}}\index{suffix?\tdescr{suffix}}%
\index{tertiary?\tdescr{tertiary}}\index{tertiary binop?\tdescr{tertiary binop}}\index{unary op?\tdescr{unary op}}%
\label{syexpr1}
\end{figure}

\begin{figure}[htp]
\begin{ctabbing}
$\tt \descr{transformer} \rightarrow rotated \descr{numeric primary}$\\
$\tt \qquad \;|\; scaled \descr{numeric primary}$\\
$\tt \qquad \;|\; shifted \descr{pair primary}$\\
$\tt \qquad \;|\; slanted \descr{numeric primary}$\\
$\tt \qquad \;|\; transformed \descr{transform primary}$\\
$\tt \qquad \;|\; xscaled \descr{numeric primary}$\\
$\tt \qquad \;|\; yscaled \descr{numeric primary}$\\
$\tt \qquad \;|\; zscaled \descr{pair primary}$\\
$\tt \qquad \;|\; reflectedabout\hbox{\tt (}\descr{pair expression}\hbox{\tt ,} \descr{pair expression}\hbox{\tt )}$\\
$\tt \qquad \;|\; rotatedaround\hbox{\tt (}\descr{pair expression}\hbox{\tt ,} \descr{numeric expression}\hbox{\tt )}$\\
$\tt $\\
$\tt \descr{nullary op} \rightarrow false \;|\; normaldeviate \;|\; nullpen \;|\; nullpicture \;|\; pencircle$\\
$\tt \qquad \;|\; true \;|\; whatever$\\
$\tt \descr{unary op} \rightarrow \descr{type}$\\
$\tt \qquad \;|\; abs \;|\; angle \;|\; arclength \;|\; ASCII \;|\; bbox \;|\; blackpart \;|\; bluepart \;|\; bot \;|\; bounded$\\
$\tt \qquad \;|\; ceiling \;|\; center \;|\; char \;|\; clipped \;|\; colormodel \;|\; cosd \;|\; cyanpart  \;|\; cycle$\\
$\tt \qquad \;|\; dashpart \;|\; decimal \;|\; dir \;|\; floor \;|\; filled \;|\; fontpart \;|\; fontsize$\\
$\tt \qquad \;|\; greenpart \;|\; greypart \;|\; hex \;|\; inverse \;|\; known \;|\; length \;|\; lft \;|\; llcorner$\\
$\tt \qquad \;|\; lrcorner\;|\; magentapart \;|\; makepath \;|\; makepen \;|\; mexp \;|\; mlog \;|\; not \;|\; oct \;|\; odd$\\
$\tt \qquad \;|\; pathpart \;|\; penpart \;|\; readfrom \;|\; redpart \;|\; reverse \;|\; round \;|\; rt \;|\; sind \;|\; sqrt$\\
$\tt \qquad  \;|\; stroked \;|\; textpart \;|\; textual \;|\; top \;|\; ulcorner$\\
$\tt \qquad \;|\; uniformdeviate \;|\; unitvector \;|\; unknown \;|\; urcorner \;|\; xpart \;|\; xxpart$\\
$\tt \qquad \;|\; xypart \;|\; yellowpart \;|\; ypart \;|\; yxpart \;|\; yypart$\\
$\tt \descr{type} \rightarrow boolean \;|\; cmykcolor \;|\; color \;|\; numeric \;|\; pair$\\
$\tt \qquad \;|\; path \;|\; pen \;|\; picture \;|\; rgbcolor \;|\; string \;|\; transform$\\
$\tt \descr{internal type} \rightarrow numeric \;|\; string$\\
$\tt \descr{primary binop} \rightarrow \hbox{\tt *} \;|\; \hbox{\tt /} \;|\; \hbox{\tt **} \;|\; and$\\
$\tt \qquad \;|\; dotprod \;|\; div \;|\; infont \;|\; mod$\\
$\tt \descr{secondary binop} \rightarrow + \;|\; - \;|\; ++ \;|\; +-+ \;|\; or$\\
$\tt \qquad \;|\; intersectionpoint \;|\; intersectiontimes$\\
$\tt \descr{tertiary binop} \rightarrow \hbox{\tt \&} \;|\; \hbox{\verb|<|} \;|\; \hbox{\verb|<=|} \;|\; \hbox{\verb|<>|} \;|\; \hbox{\tt =} \;|\; \hbox{\verb|>|} \;|\; \hbox{\verb|>=|}$\\
$\tt \qquad \;|\; cutafter \;|\; cutbefore$\\
$\tt \descr{of operator} \rightarrow arctime \;|\; direction \;|\; directiontime \;|\; directionpoint$\\
$\tt \qquad \;|\; glyph \;|\; penoffset \;|\; point \;|\; postcontrol \;|\; precontrol$\\
$\tt \qquad \;|\; subpath \;|\; substring$\\
$\tt $\\
$\tt \descr{variable} \rightarrow \descr{tag}\descr{suffix}$\\
$\tt \descr{suffix} \rightarrow \descr{empty} \;|\; \descr{suffix}\descr{subscript} \;|\; \descr{suffix}\descr{tag}$\\
$\tt \qquad \;|\; \descr{suffix parameter}$\\
$\tt \descr{subscript} \rightarrow \descr{number} \;|\; \hbox{\tt [}\descr{numeric expression}\hbox{\tt ]}$\\
$\tt $\\
$\tt \descr{internal variable} \rightarrow ahangle \;|\; ahlength \;|\; bboxmargin$\\
$\tt \qquad \;|\; charcode \;|\; day \;|\; defaultcolormodel \;|\; defaultpen \;|\; defaultscale$\\
$\tt \qquad \;|\; hour \;|\; hppp \;|\; jobname \;|\; labeloffset \;|\; linecap \;|\; linejoin
\;|\; minute \;|\; miterlimit \;|\; month$\\
$\tt \qquad \;|\; numberprecision \;|\; numbersystem$\\
$\tt \qquad \;|\; outputfilename \;|\; outputformat \;|\; outputformatoptions \;|\; outputtemplate$\\
$\tt \qquad \;|\; pausing \;|\; prologues \;|\; showstopping$\\
$\tt \qquad \;|\; time \;|\; tracingoutput \;|\; tracingcapsules \;|\; tracingchoices \;|\; tracingcommands$\\
$\tt \qquad \;|\; tracingequations \;|\; tracinglostchars \;|\; tracingmacros$\\
$\tt \qquad \;|\; tracingonline \;|\; tracingrestores \;|\; tracingspecs$\\
$\tt \qquad \;|\; tracingstats \;|\; tracingtitles \;|\; truecorners$\\
$\tt \qquad \;|\; warningcheck \;|\; vppp \;|\; year$\\
$\tt \qquad \;|\; \descr{symbolic token defined by {\tt newinternal}}$
\end{ctabbing}
\caption{Part 2 of the syntax for expressions}
\index{nullary op?\tdescr{nullary op}}\index{of operator?\tdescr{of operator}}\index{primary binop?\tdescr{primary binop}}%
\index{secondary binop?\tdescr{secondary binop}}\index{subscript?\tdescr{subscript}}\index{suffix?\tdescr{suffix}}%
\index{tertiary binop?\tdescr{tertiary binop}}\index{unary op?\tdescr{unary op}}%
\label{syexpr2}
\end{figure}

\begin{figure}[htp]
\begin{ctabbing}
$\tt \descr{pseudo function} \rightarrow min\hbox{\tt (}\descr{expression list}\hbox{\tt )}$\\
$\tt \qquad \;|\; max\hbox{\tt (}\descr{expression list}\hbox{\tt )}$\\
$\tt \qquad \;|\; incr\hbox{\tt (}\descr{numeric variable}\hbox{\tt )}$\\
$\tt \qquad \;|\; decr\hbox{\tt (}\descr{numeric variable}\hbox{\tt )}$\\
$\tt \qquad \;|\; dashpattern\hbox{\tt (}\descr{on\hbox{\tt /}off list}\hbox{\tt )}$\\
$\tt \qquad \;|\; interpath\hbox{\tt (}\descr{numeric expression}\hbox{\tt ,} \descr{path expression}\hbox{\tt ,} \descr{path expression}\hbox{\tt )}$\\
$\tt \qquad \;|\; buildcycle\hbox{\tt (}\descr{path expression list}\hbox{\tt )}$\\
$\tt \qquad \;|\; thelabel\descr{label suffix}\hbox{\tt (}\descr{expression}\hbox{\tt ,} \descr{pair expression}\hbox{\tt )}$\\
$\tt \descr{path expression list} \rightarrow \descr{path expression}$\\
$\tt \qquad \;|\; \descr{path expression list}\hbox{\tt ,} \descr{path expression}$\\
$\tt \descr{on\hbox{\tt /}off list} \rightarrow \descr{on\hbox{\tt /}off list}\descr{on\hbox{\tt /}off clause} \;|\; \descr{on\hbox{\tt /}off clause}$\\
$\tt \descr{on\hbox{\tt /}off clause} \rightarrow on \descr{numeric tertiary} \;|\; off \descr{numeric tertiary}$
\end{ctabbing}
\caption{The syntax for function-like macros}
\index{label suffix?\tdescr{label suffix}}%
\label{sypseudo}
\end{figure}

\begin{figure}[htp]
\begin{ctabbing}
$\tt \descr{boolean expression} \rightarrow \descr{expression}$\\
$\tt \descr{cmykcolor expression} \rightarrow \descr{expression}$\\
$\tt \descr{color expression} \rightarrow \descr{expression}$\\
$\tt \descr{numeric atom} \rightarrow \descr{atom}$\\
$\tt \descr{numeric expression} \rightarrow \descr{expression}$\\
$\tt \descr{numeric primary} \rightarrow \descr{primary}$\\
$\tt \descr{numeric tertiary} \rightarrow \descr{tertiary}$\\
$\tt \descr{numeric variable} \rightarrow \descr{variable} \;|\; \descr{internal variable}$\\
$\tt \descr{pair expression} \rightarrow \descr{expression}$\\
$\tt \descr{pair primary} \rightarrow \descr{primary}$\\
$\tt \descr{path expression} \rightarrow \descr{expression}$\\
$\tt \descr{path subexpression} \rightarrow \descr{subexpression}$\\
$\tt \descr{pen expression} \rightarrow \descr{expression}$\\
$\tt \descr{picture expression} \rightarrow \descr{expression}$\\
$\tt \descr{picture variable} \rightarrow \descr{variable}$\\
$\tt \descr{rgbcolor expression} \rightarrow \descr{expression}$\\
$\tt \descr{string expression} \rightarrow \descr{expression}$\\
$\tt \descr{suffix parameter} \rightarrow \descr{parameter}$\\
$\tt \descr{transform primary} \rightarrow \descr{primary}$
\end{ctabbing}
\caption{Miscellaneous productions needed to complete the BNF}
\label{sytypexpr}
\end{figure}

\begin{figure}[htp]
\begin{ctabbing}
$\tt \descr{program} \rightarrow \descr{statement list} end$\\
$\tt \descr{statement list} \rightarrow \descr{empty} \;|\; \descr{statement list} \hbox{\tt ;} \descr{statement}$\\
$\tt \descr{statement} \rightarrow \descr{empty}$\\
$\tt \qquad \;|\; \descr{equation} \;|\; \descr{assignment}$\\
$\tt \qquad \;|\; \descr{declaration} \;|\; \descr{macro definition}$\\
$\tt \qquad \;|\; \descr{compound} \;|\; \descr{pseudo procedure}$\\
$\tt \qquad \;|\; \descr{command}$\\
$\tt \descr{compound} \rightarrow begingroup \descr{statement list} endgroup$\\
$\tt \qquad \;|\; beginfig\hbox{\tt (}\descr{numeric expression}\hbox{\tt );} \descr{statement list}\hbox{\tt ;} endfig$\\
$\tt $\\
$\tt \descr{equation} \rightarrow \descr{expression} \hbox{\tt =} \descr{right-hand side}$\\
$\tt \descr{assignment} \rightarrow \descr{variable} \hbox{\tt :=} \descr{right-hand side}$\\
$\tt \qquad \;|\; \descr{internal variable} \hbox{\tt :=} \descr{right-hand side}$\\
$\tt \descr{right-hand side} \rightarrow \descr{expression} \;|\; \descr{equation} \;|\; \descr{assignment}$\\
$\tt $\\
$\tt \descr{declaration} \rightarrow \descr{type} \descr{declaration list}$\\
$\tt \descr{declaration list} \rightarrow \descr{generic variable}$\\
$\tt \qquad \;|\; \descr{declaration list}\hbox{\tt ,} \descr{generic variable}$\\
$\tt \descr{generic variable} \rightarrow \descr{symbolic token} \descr{generic suffix}$\\
$\tt \descr{generic suffix} \rightarrow \descr{empty} \;|\; \descr{generic suffix} \descr{tag}$\\
$\tt \qquad \;|\; \descr{generic suffix} \hbox{\tt []}$\\
$\tt $\\
$\tt \descr{macro definition} \rightarrow \descr{macro heading} \hbox{\tt =} \descr{replacement text} enddef$\\
$\tt \descr{macro heading} \rightarrow def \descr{symbolic token} \descr{delimited part} \descr{undelimited part}$\\
$\tt \qquad \;|\; vardef \descr{generic variable} \descr{delimited part} \descr{undelimited part}$\\
$\tt \qquad \;|\; vardef \descr{generic variable} \hbox{\verb|@#|} \descr{delimited part} \descr{undelimited part}$\\
$\tt \qquad \;|\; \descr{binary def} \descr{parameter} \descr{symbolic token} \descr{parameter}$\\
$\tt \descr{delimited part} \rightarrow \descr{empty}$\\
$\tt \qquad \;|\; \descr{delimited part}\hbox{\tt (}\descr{parameter type} \descr{parameter tokens}\hbox{\tt )}$\\
$\tt \descr{parameter type} \rightarrow expr \;|\; suffix \;|\; text$\\
$\tt \descr{parameter tokens} \rightarrow \descr{parameter} \;|\; \descr{parameter tokens}\hbox{\tt ,} \descr{parameter}$\\
$\tt \descr{parameter} \rightarrow \descr{symbolic token}$\\
$\tt \descr{undelimited part} \rightarrow \descr{empty}$\\
$\tt \qquad \;|\; \descr{parameter type} \descr{parameter}$\\
$\tt \qquad \;|\; \descr{precedence level} \descr{parameter}$\\
$\tt \qquad \;|\; expr \descr{parameter} of \descr{parameter}$\\
$\tt \descr{precedence level} \rightarrow primary \;|\; secondary \;|\; tertiary$\\
$\tt \descr{binary def} \rightarrow primarydef \;|\; secondarydef \;|\; tertiarydef$\\
$\tt $\\
$\tt \descr{pseudo procedure} \rightarrow drawoptions\hbox{\tt (}\descr{option list}\hbox{\tt )}$\\
$\tt \qquad \;|\; label\descr{label suffix}\hbox{\tt (}\descr{expression}\hbox{\tt ,} \descr{pair expression}\hbox{\tt )}$\\
$\tt \qquad \;|\; dotlabel\descr{label suffix}\hbox{\tt (}\descr{expression}\hbox{\tt ,} \descr{pair expression}\hbox{\tt )}$\\
$\tt \qquad \;|\; labels\descr{label suffix}\hbox{\tt (}\descr{point number list}\hbox{\tt )}$\\
$\tt \qquad \;|\; dotlabels\descr{label suffix}\hbox{\tt (}\descr{point number list}\hbox{\tt )}$\\
$\tt \descr{point number list} \rightarrow \descr{suffix} \;|\; \descr{point number list}\hbox{\tt ,} \descr{suffix}$\\
$\tt \descr{label suffix} \rightarrow \descr{empty} \;|\; lft \;|\; rt \;|\; top \;|\; bot \;|\; ulft \;|\; urt \;|\; llft \;|\; lrt$
\end{ctabbing}
\caption{Overall syntax for MetaPost programs}
\index{generic variable?\tdescr{generic variable}}\index{label suffix?\tdescr{label suffix}}\index{replacement text?\tdescr{replacement text}}%
\index{suffix?\tdescr{suffix}}%
\label{syprog}
\end{figure}

\begin{figure}[htp]
\begin{ctabbing}
$\tt \descr{command} \rightarrow clip \descr{picture variable} to \descr{path expression}$\\
$\tt \qquad \;|\; interim \descr{internal variable} \hbox{\tt :=} \descr{right-hand side}$\\
$\tt \qquad \;|\; let \descr{symbolic token} \hbox{\tt =} \descr{symbolic token}$\\
$\tt \qquad \;|\; pickup \descr{expression}$\\
$\tt \qquad \;|\; randomseed \hbox{\tt :=} \descr{numeric expression}$\\
$\tt \qquad \;|\; save \descr{symbolic token list}$\\
$\tt \qquad \;|\; setbounds \descr{picture variable} to \descr{path expression}$\\
$\tt \qquad \;|\; shipout \descr{picture expression}$\\
$\tt \qquad \;|\; write \descr{string expression} to \descr{string expression}$\\
$\tt \qquad \;|\; \descr{addto command}$\\
$\tt \qquad \;|\; \descr{drawing command}$\\
$\tt \qquad \;|\; \descr{font metric command}$\\
$\tt \qquad \;|\; \descr{newinternal command}$\\
$\tt \qquad \;|\; \descr{message command}$\\
$\tt \qquad \;|\; \descr{mode command}$\\
$\tt \qquad \;|\; \descr{show command}$\\
$\tt \qquad \;|\; \descr{special command}$\\
$\tt \qquad \;|\; \descr{tracing command}$\\
$\tt $\\
$\tt \descr{show command} \rightarrow show \descr{expression list}$\\
$\tt \qquad \;|\; showvariable \descr{symbolic token list}$\\
$\tt \qquad \;|\; showtoken \descr{symbolic token list}$\\
$\tt \qquad \;|\; showdependencies$\\
$\tt $\\
$\tt \descr{symbolic token list} \rightarrow \descr{symbolic token}$\\
$\tt \qquad \;|\; \descr{symbolic token}\hbox{\tt ,} \descr{symbolic token list}$\\
$\tt \descr{expression list} \rightarrow \descr{expression} \;|\; \descr{expression list}\hbox{\tt ,} \descr{expression}$\\
$\tt $\\
$\tt \descr{addto command} \rightarrow$\\
$\tt \qquad addto \descr{picture variable} also \descr{picture expression} \descr{option list}$\\
$\tt \qquad \;|\; addto \descr{picture variable} contour \descr{path expression} \descr{option list}$\\
$\tt \qquad \;|\; addto \descr{picture variable} doublepath \descr{path expression} \descr{option list}$\\
$\tt \descr{option list} \rightarrow \descr{empty} \;|\; \descr{drawing option} \descr{option list}$\\
$\tt \descr{drawing option} \rightarrow withcolor \descr{color expression}$\\
$\tt \qquad \;|\; withrgbcolor \descr{rgbcolor expression} \;|\; withcmykcolor \descr{cmykcolor expression}$\\
$\tt \qquad \;|\; withgreyscale \descr{numeric expression} \;|\; withoutcolor$\\
$\tt \qquad \;|\; withprescript \descr{string expression} \;|\; withpostscript \descr{string expression}$\\
$\tt \qquad \;|\; withpen \descr{pen expression} \;|\; dashed \descr{picture expression}$\\
$\tt $\\
$\tt \descr{drawing command} \rightarrow draw \descr{picture expression} \descr{option list}$\\
$\tt \qquad \;|\; \descr{fill type} \descr{path expression} \descr{option list}$\\
$\tt \descr{fill type} \rightarrow fill \;|\; draw \;|\; filldraw \;|\; unfill \;|\; undraw \;|\; unfilldraw$\\
$\tt \qquad \;|\; drawarrow \;|\; drawdblarrow \;|\; cutdraw$\\
$\tt $\\
$\tt \descr{newinternal command} \rightarrow newinternal \descr{internal type} \descr{symbolic token list}$\\
$\tt \qquad \;|\; newinternal \descr{symbolic token list}$\\
$\tt $\\
$\tt \descr{message command} \rightarrow errhelp \descr{string expression}$\\
$\tt \qquad \;|\; errmessage \descr{string expression}$\\
$\tt \qquad \;|\; filenametemplate \descr{string expression}$\\
$\tt \qquad \;|\; message \descr{string expression}$
\end{ctabbing}
\caption{Part 1 of the syntax for commands}
\index{option list?\tdescr{option list}}\index{picture variable?\tdescr{picture variable}}%
\label{sycmds1}

% is this really true?  in practice, in real implementations?
%\bigskip
%By the way, the default for \ttindex{randomseed} is, in effect,
%$\textsl{day} + \textsl{time}*\epsilon$.  Unfortunately, \textsl{time}
%is in minutes.
\end{figure}

\begin{figure}[htp]
\begin{ctabbing}
$\tt \descr{mode command} \rightarrow batchmode \;|\; nonstopmode$\\
$\tt \qquad \;|\; scrollmode \;|\; errorstopmode$\\
$\tt $\\
$\tt \descr{special command} \rightarrow fontmapfile \descr{string expression}$\\
$\tt \qquad \;|\; fontmapline \descr{string expression}$\\
$\tt \qquad \;|\; special \descr{string expression}$\\
$\tt $\\
$\tt \descr{tracing command} \rightarrow tracingall \;|\; loggingall \;|\; tracingnone$
\end{ctabbing}
\caption{Part 2 of the syntax for commands}
\index{batchmode?\texttt{batchmode}}\index{nonstopmode?\texttt{nonstopmode}}\index{scrollmode?\texttt{scrollmode}}\index{errorstopmode?\texttt{errorstopmode}}%
\label{sycmds2}
\end{figure}

\begin{figure}[htp]
\begin{ctabbing}
$\tt \descr{if test} \rightarrow if \descr{boolean expression} \hbox{\tt :} \descr{balanced tokens} \descr{alternatives} fi$\\
$\tt \descr{alternatives} \rightarrow \descr{empty}$\\
$\tt \qquad \;|\; else\hbox{\tt :} \descr{balanced tokens}$\\
$\tt \qquad \;|\; elseif \descr{boolean expression} \hbox{\tt :} \descr{balanced tokens} \descr{alternatives}$\\
$\tt $\\
$\tt \descr{loop} \rightarrow \descr{loop header}\hbox{\tt :} \descr{loop text} endfor$\\
$\tt \descr{loop header} \rightarrow for \descr{symbolic token} \hbox{\tt =} \descr{progression}$\\
$\tt \qquad \;|\; for \descr{symbolic token} \hbox{\tt =} \descr{for list}$\\
$\tt \qquad \;|\; for \descr{symbolic token} \hbox{\tt within} \descr{picture expression}$\\
$\tt \qquad \;|\; forsuffixes \descr{symbolic token} \hbox{\tt =} \descr{suffix list}$\\
$\tt \qquad \;|\; forever$\\
$\tt \descr{progression} \rightarrow \descr{numeric expression} upto \descr{numeric expression}$\\
$\tt \qquad \;|\; \descr{numeric expression} downto \descr{numeric expression}$\\
$\tt \qquad \;|\; \descr{numeric expression} step \descr{numeric expression} until \descr{numeric expression} $\\
$\tt \descr{for list} \rightarrow \descr{expression} \;|\; \descr{for list}\hbox{\tt ,} \descr{expression}$\\
$\tt \descr{suffix list} \rightarrow \descr{suffix} \;|\; \descr{suffix list}\hbox{\tt ,} \descr{suffix}$
\end{ctabbing}
\caption{The syntax for conditionals and loops}
\index{balanced tokens?\tdescr{balanced tokens}}\index{suffix?\tdescr{suffix}}%
\label{sycondloop}
\end{figure}

\endgroup


\subsection{Command-Line Syntax}
\label{refman:commandline}

\begingroup
\renewcommand*{\arraystretch}{1.25}% Give command-line switch tables air.

\subsubsection{The MetaPost Program}
\label{refman:mpost}

\newcommand*{\opt}[1]{\ensuremath{\big[}#1\ensuremath{\big]}}
\newcommand*{\cmdsw}{\tdescr{switches}}
\newcommand*{\cmdmem}{\tdescr{preloadfile}}
\newcommand*{\cmdin}{\tdescr{infile}}
\newcommand*{\cmdcode}{\tdescr{commands}}
\newcommand*{\cmdindex}[1]{\index{command-line!mpost?\texttt{mpost}!#1?\texttt{#1}}\texttt{#1}}

The MetaPost program processes commands in the MetaPost language, either
read from a file or typed-in interactively, and compiles them into
PostScript or SVG graphics.  The run-time behavior of the executable can
be controlled by command-line parameters, environment variables, and
configuration files.  The MetaPost executable is named
mpost\label{Dmpost}\index{mpost} (or occasionally just mp).  The
command-line syntax of the executable is
$$
\begin{verbatim}[commandchars=\\\{\}]
mpost \opt{\cmdsw} \opt{\&\cmdmem} \opt{\cmdin} \opt{\cmdcode}
\end{verbatim}
\index{mpost?\texttt{mpost}}\index{command-line!mpost?\texttt{mpost}}
$$
Any of the parameters is optional.  If no argument is given to the
\verb|mpost| command, MetaPost enters interactive mode\index{interactive
  mode}, indicated by a \verb|**| prompt\index{**?\texttt{**}!prompt},
waiting for a file name to be typed-in by keyboard.  The file is then
read-in and processed as if it were given as parameter \cmdin\ on the
command-line.

\begin{description}
\item[\cmdin] A MetaPost input file is a text file containing statements
  in the MetaPost language.  Typically, input files have file extension
  \texttt{mp}\index{mp file?{\tt mp} file}\index{files!mp?{\tt mp}},
  e.g., \texttt{fig.mp}.  Here is how MetaPost searches for input files.
  If \cmdin\ doesn't end with \texttt{.mp}\index{mp file?{\tt mp}
    file}\index{files!mp?{\tt mp}}, MetaPost first looks for a file
  \texttt{\cmdin.mp}.  Only if that file doesn't exist, it looks for
  file \cmdin\ literally.  That way, the \texttt{mp}\index{mp file?{\tt
      mp} file}\index{files!mp?{\tt mp}} file extension can be omitted
  when calling MetaPost.  If \cmdin\ already ends with
  \texttt{.mp}\index{mp file?{\tt mp} file}\index{files!mp?{\tt mp}}, no
  special file name handling takes place and MetaPost looks for that
  file only.

  As an example, if there are two files \texttt{fig} and \texttt{fig.mp}
  in the current directory and MetaPost is invoked with
  \verb|'mpost fig'|, the file that gets processed is \texttt{fig.mp}.
  To process file \texttt{fig} in this situation, MetaPost can be called
  as \verb|'mpost fig.'|.  Note, the trailing dot is only needed as long
  as there exists a file \texttt{fig.mp} alongside file \texttt{fig}.

  If MetaPost cannot find any input file by the rules specified above,
  it complaints with an error message and interactively asks for a new
  input file name.

\item[\cmdcode] All text on the command-line after \cmdin\ is
  interpreted as MetaPost code and is processed after \cmdin\ is read.
  If \cmdin\ already contains an \verb|end| statement, \cmdcode\ gets
  effectively ignored.  If MetaPost doesn't encounter an \verb|end|
  statement neither in \cmdin\ nor in \cmdcode, it enters interactive
  mode after processing all input.

\item[\texttt{\&\cmdmem}] The MetaPost program code in the \cmdmem\
  MetaPost source file (and any other files that it inputs) is executed
  before all other processing starts. The \cmdmem\ file should end with
  an \texttt{end}\index{end?\texttt{end}} or
  \texttt{dump}\index{dump?\texttt{dump}} command.

  If neither \texttt{\&\cmdmem} nor \cmdindex{-mem} is specified on the
  command line, MetaPost will default to loading the macro file
  \texttt{mpost.mp}\index{mpost.mp?\texttt{mpost.mp}}, or to whatever
  the actual name of the MetaPost executable is. The \cmdindex{-ini}
  switch can be used to suppress this behavior.

\item[\cmdsw] MetaPost provides a number of command-line switches that
  control the run-time behavior.  Switches can be prefixed by one or two
  dashes.  Both forms have the same meaning.  An exemplary call to
  MetaPost that compiles a file \texttt{fig.mp}, using \LaTeX\ to
  typeset \verb|btex|\slash\verb|etex| labels, would look like:
$$
\begin{verbatim}
mpost -tex=latex fig
\end{verbatim}
\index{mpost?\texttt{mpost}}
$$

Here's a summary of the command-line switches provided by mpost:

\setlength{\LTleft}{\leftmargin}
\begin{longtable}{>{\ttfamily}p{.35\linewidth}>{\raggedright\arraybackslash}p{.6\linewidth}}
  \cmdindex{-debug}
  & Don't delete intermediate files.\\
  \cmdindex{-dvitomp}
  & Act as the \texttt{dvitomp} executable (see below).\\
  \cmdindex{-file-line-error}
  & Start error messages with \verb|`filename:lineno:'| instead of
  \verb|`!'|.\\
  \cmdindex{-halt-on-error}
  & Immediately exit after the first error occurred.\\
  \cmdindex{-help}
  & Show help on command-line switches.\\
  \cmdindex{-ini}
  & Do not load any preload file.\\
  \cmdindex{-interaction}=\tdescr{string}
  & Set interaction mode to one of \texttt{batchmode},
  \texttt{nonstopmode}, \texttt{scrollmode}, \texttt{errorstopmode}.\\
  \cmdindex{-jobname}=\tdescr{jobname}
  & Set the name of the job (affects output file names).\\
  \cmdindex{-kpathsea-debug}=\tdescr{number}
  & Set debugging flags for path searching.\\
  \cmdindex{-mem}=\tdescr{string}
  & Use \tdescr{string} for the name of the file that contains macros to be preloaded 
  (same as \texttt{\&\cmdmem})\\
  \cmdindex{-no-file-line-error}
  & Enable normal MetaPost and \TeX\ style error messages.\\
  \cmdindex{-no-kpathsea}
  & Do not use the kpathsea program to find files.  All files have to be
  in the current directory or specified via a full path.\\
  \cmdindex{-numbersystem}=\tdescr{string}
  & Set arithmetic mode to one of \texttt{scaled}, \texttt{double},
  \texttt{binary}, \texttt{decimal}\\
  \cmdindex{-progname}=\tdescr{string}
  & Pretend to be the \tdescr{string} executable.\\
  \cmdindex{-recorder}
  & Write a list of all opened disk files to
  \texttt{\tdescr{jobname}.fls}\index{fls file?{\tt fls}
    file}\index{files!fls?{\tt fls}}.  (This functionality is provided
  by kpathsea.)\\
  \cmdindex{-s} \tdescr{key}=\tdescr{value}
  & Set internal variable \tdescr{key} to \tdescr{value}.  This switch
  can be given multiple times on the command-line.  The assignments are
  applied just before the input file is read-in.  \tdescr{value} can be
  an integer between -16383 and 16383 or a string in double quotes.  For
  strings, double quotes are stripped, but no other processing takes
  place.  To avoid double quotes being already stripped by the shell,
  the whole assignment can be enclosed in another pair of single quotes.
  Example: \verb|-s 'outputformat="svg"' -s prologues=3|\newline
  Use SVG backend converting font shapes to paths.\\
  \cmdindex{-tex}=\tdescr{texprogram}
  & Load format \tdescr{texprogram} for rendering \TeX\ material.\\
  \cmdindex{-troff}, \cmdindex{-T}
  & Output troff compatible PostScript files.\\
  \cmdindex{-version}
  & Print version information and exit.\\
\end{longtable}

The following command-line switches are silently ignored in
\emph{mplib}-based MetaPost (v1.100 or later), because they are always
`on':

\begin{longtable}{>{\ttfamily}p{.35\linewidth}>{\raggedright\arraybackslash}p{.6\linewidth}}
  \cmdindex{-8bit}
  & \\
  \cmdindex{-parse-first-line}
  & \\
\end{longtable}

The following command-line switches are ignored, but trigger a warning:

\begin{longtable}{>{\ttfamily}p{.35\linewidth}>{\raggedright\arraybackslash}p{.6\linewidth}}
  \cmdindex{-no-parse-first-line}
  & \\
  \cmdindex{-output-directory}=\tdescr{string}
  & \\
  \cmdindex{-translate-file}=\tdescr{string}
  & \\
\end{longtable}

\end{description}

\subsubsection{The dvitomp Program}
\label{refman:dvitomp}

\renewcommand*{\cmdindex}[1]{\index{command-line!dvitomp?\texttt{dvitomp}!#1?\texttt{#1}}\texttt{#1}}
\renewcommand*{\cmdin}{\tdescr{infile}}
\newcommand*{\cmdout}{\tdescr{outfile}}

The dvitomp\label{Ddvitomp}\index{dvitomp} program converts
DVI\index{dvi file?{\tt dvi} file}\index{files!dvi?{\tt dvi}} files into
low-level MetaPost code.  MetaPost uses the dvitomp program when
typesetting \verb|btex|\slash\verb|etex| labels by \TeX\ for the final
conversion back into MetaPost code.  The command-line syntax of the
executable is
$$
\begin{verbatim}[commandchars=\\\{\}]
dvitomp \opt{\cmdsw} \cmdin \opt{\cmdout}
\end{verbatim}
\index{dvitomp?\texttt{dvitomp}}\index{command-line!dvitomp?\texttt{dvitomp}}
$$
Parameters \cmdsw\ and \cmdout\ are optional.

\begin{description}
\item[\cmdin] This is the name of the DVI file to convert.  If the name
  doesn't end with \texttt{.dvi}\index{dvi file?{\tt dvi}
    file}\index{files!dvi?{\tt dvi}}, that extension is appended.  Note,
  dvitomp never opens files not ending \texttt{.dvi}.  This file is in
  general automatically generated by \TeX\ or troff, driven by MetaPost.

\item[\cmdout] This is the name of the output file containing the
  MetaPost code equivalent to \cmdin.  If \cmdout\ is not given,
  \cmdout\ is \cmdin\ with the extension \texttt{.dvi} replaced by
  \texttt{.mpx}\index{mpx file?{\tt mpx} file}\index{files!mpx?{\tt
      mpx}}.  If \cmdout\ is given and doesn't end with
  \texttt{.mpx}\index{mpx file?{\tt mpx} file}\index{files!mpx?{\tt
      mpx}}, that extension is appended.

\item[\cmdsw] Command-line switches can be prefixed by one or two
  dashes.  Both forms have the same meaning.  The following command-line
  switches are provided by dvitomp:

\setlength{\LTleft}{\leftmargin}
\begin{longtable}{>{\ttfamily}p{.35\linewidth}>{\raggedright\arraybackslash}p{.6\linewidth}}
  \cmdindex{-help}
  & Show help on command-line switches.\\
  \cmdindex{-kpathsea-debug}=\tdescr{number}
  & Set debugging flags for path searching.\\
  \cmdindex{-progname}=\tdescr{string}
  & Pretend to be the \tdescr{string} executable.\\
  \cmdindex{-version}
  & Print version information and exit.\\
\end{longtable}

\end{description}

The dvitomp program used to be part of a set of external tools, called
\emph{mpware}\index{mpware tools}\footnote{makempx, mpto, dvitomp, and
  dmp.}, which were used by MetaPost for processing
\verb|btex|\slash\verb|etex| labels.  Since MetaPost version~1.100, the
label conversion is handled internally by the mpost\index{mpost}
program.  The \emph{mpware} tools are therefore obsolete and no longer
part of the MetaPost distribution.  Nowadays, dvitomp is either a copy
of the mpost executable with the name dvitomp or a wrapper, calling
mpost\index{mpost} as
$$
\begin{verbatim}[commandchars=\\\{\}]
mpost -dvitomp \cmdin \cmdout
\end{verbatim}
\index{mpost?\texttt{mpost}}
$$

\endgroup

%%% Local Variables: 
%%% mode: latex
%%% TeX-PDF-mode: t
%%% TeX-master: "mpman"
%%% End: 
