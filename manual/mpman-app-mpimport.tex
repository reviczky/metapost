\svnInfo $Id$
\section{Importing MetaPost graphics in \TeX\ documents}
\label{teximport}
\index{TeX?\TeX!importing MetaPost files}
\index{LaTeX?\LaTeX!importing MetaPost files}
\index{pdfLaTeX?pdf\LaTeX!importing MetaPost files}
\index{pdfTeX?pdf\TeX!importing MetaPost files}
\index{ConTeXt?Con\TeX t!importing MetaPost files}

MetaPost graphics in the PostScript format can be easily integreated
into documents prepared with \TeX\ and friends.  MetaPost's PostScript
output is a low-featured dialect of the Postscript language, called
\emph{purified EPS}, that can be converted into the Portable Document
Format (PDF\index{PDF}) language on-the-fly.  For that reason, MetaPost
graphics can be handled by both engines, the traditional \TeX\ with an
external output driver and the newer pdf\TeX, that contains a built-in
PDF output driver.

Figure~\ref{fig0} shows the process of including a MetaPost graphic into
a plain \TeX\ or \LaTeX\ document.  In the \TeX\ document a ``magic
macro'' provided by the format or an external package is used for
including a graphic file.  During the typesetting stage, the macro only
reads bounding box information off the PostScript file and reserves the
required space on the page via an empty box.  The file reference is
passed-on to the output driver and only then, finally, the file is
embedded into the document.  The freely available program
\ttindex{dvips} is used as an output driver in this
example.\footnote{The C source for \ttt{dvips} comes with the web2c
  \TeX\ distribution.  Similar programs are available from other
  sources.}  The next paragraphs give brief information on some popular
combinations of formats and engines.

\begin{figure}[htp]
$$ \includegraphics{mpman-charts-1.mps} $$
\caption[A diagram of the processing for a document with MetaPost figures]
        {A diagram of the processing for a \TeX\ document with figures
        in MetaPost}
\label{fig0}
\end{figure}

\paragraph{Plain \TeX\ with \TeX}
For users of \TeX\ with traditional DVI output the
\ttt{epsf}\index{epsf.tex?\texttt{epsf.tex}} package provides the
``magic macro''
$$ \verb|\epsfbox{|\descr{filename}\verb|}|%
\index{epsfbox?\texttt{\string\epsfbox}} $$
for embedding graphics, e.g., \verb|\epsfbox{fig.1}|.

\paragraph{Plain \TeX\ with pdf\TeX}
Users of Plain \TeX\ and pdf\TeX\ should refer to the standalone macros
of the \ttindex{mptopdf} bundle, that can be found at
\url{http://context.aanhet.net/mptopdf.htm}.

\paragraph{\LaTeX\ format}
For \LaTeX\ documents the well-known \ttindex{graphics} (or
\ttindex{graphicx}) package aids in external graphics inclusion.  The
package supports different engines and can handle several graphic
formats.  The ``magic macro'' is
$$ \verb|\includegraphics{|\descr{filename}\verb|}|%
\index{includegraphics?\texttt{\string\includegraphics}} $$

When the \ttt{dvips} driver is chosen, the \ttt{graphics} package
assumes all files with an unknown file extension to be in the EPS
format.  MetaPost files with a numeric default file extension are
therefore handled correctly -- even if only in a fall-back procedure
(see~\cite{reckdahl:epslatex} for more information).

For the pdf\TeX\ driver, the situation is a bit different.  Only files
with file extension \texttt{.mps}\index{mps file?{\tt mps}
  file}\index{files!mps?{\tt mps}} are recognized as purified EPS and
can be converted to PDF\index{PDF} on-the-fly.  The recommended
procedure for including MetaPost graphics into \LaTeX\ documents
compiled with the pdf\TeX\ engine should be obvious: change MetaPost's
output file naming scheme to write files ending \texttt{.mps} via
\verb|outputtemplate|\index{outputtemplate?\texttt{outputtemplate}} (see
p.~\pageref{Doutputtemplate}).  In the \LaTeX\ document include the
graphic files with full name, e.g.,
$$\begin{verbatim}
\includegraphics{fig-1.mps}
\end{verbatim}
$$

Note, the latter approach works with the \ttt{dvips} driver, too.  Even
though, again, this time \ttt{.mps} is an unknown file extension and
triggers EPS file handling in the fall-back procedure.  This property of
the \ttt{graphics} package, that comes in handy for MetaPost files, is
the reason many MetaPost source files start with the line
$$\begin{verbatim}
outputtemplate := "%j-%c.mps";
\end{verbatim}
$$

\paragraph{Con\TeX t}
In Con\TeX t\index{ConTeXt?Con\TeX t} support of MetaPost is integrated
in the kernel. Apart from inline graphics (see MetaFun\index{MetaFun}
manual), one can embed graphics explictely with the
\verb+\externalfigure+%
\index{externalfigure?\texttt{\string\externalfigure}} command.
Numbered graphics are recognized automatically, as are graphics with the
\ttt{mps} suffix.  Special features like shading, transparency, image
inclusion, color spaces and such are handled automatically.  In practice
Con\TeX t users will probably define MetaPost graphics in the document
source which has some advantages, like a more natural interfacing with
document properties, font support, and automatic processing.  Support
for MetaPost inclusion is present in the versions MkII as well as MkIV,
but the used methods are slightly different.  Future versions of MkIV
will support an even more tight integration.

\subsection{Importing MetaPost graphics in \emph{troff} documents}
\label{troffimport}
\index{troff!importing MetaPost files}

It is also possible to include MetaPost output in a GNU \emph{troff}
document.  The procedure is similar to Figure~\ref{fig0}: the
\ttt{grops} output processor includes PostScript files when they are
requested via \emph{troff}'s \ttt{\string\X} command.  The
\ttt{-mpspic}\index{mpspic?\texttt{-mpspic}} macro package provides a
command \verb|.PSPIC|\index{PSPIC?\texttt{.PSPIC}}, that does just that
when including an encapsulated PostScript file in the source code.  For
instance, the \emph{troff} command
$$ \hbox{\verb|.PSPIC fig.1|} $$
includes \ttt{fig.1}, using the natural height and width of the
image as given in the file's bounding box.

\subsection{Handling SVG files}
\label{svgimport}
\index{SVG!third-party applications}

SVG graphics can be easily embedded into HTML\index{HTML} documents with
the following code snippet:

$$\begin{verbatim}
<p>
  <object data="foo.svg" type="image/svg+xml" width="300" height="200">
  </object>
</p>
\end{verbatim}
$$

SVG files can also be imported by various interactive graphics editing
programs, for example GIMP\index{GIMP} or Inkscape\index{Inkscape}.  See
section~\ref{Dprologues} for information on font handling in SVG
graphics.


%%% Local Variables: 
%%% mode: latex
%%% TeX-PDF-mode: t
%%% TeX-master: "mpman"
%%% End: 
