\begin{section}{Graphing functions}
Among the most common types of figures for \TeX{} users are those which are the graphs of functions of a single variable.  Hobby recognized this and constructed a package to accomplish this task.  It is invoked by
\begin{lstlisting}[xleftmargin=80bp]
input graph;
\end{lstlisting}
\MP{} has the ability to construct data (i.e., ordered pairs) for graphing simple functions.  However, for more complicated functions, the data should probably be constructed using external programs such as \acro{MATLAB} (or Octave), Maple, Mathematica, Gnuplot, et. al.

A typical data file, say \textattachfile[color={0 0 0},mimetype={text/plain}]{data.d}{\texttt{data.d}}, to be used with the \texttt{graph} package may have contents
\lstinputlisting[xleftmargin=77bp]{data.d}
This data represents the graph of $f(x)=\sqrt{x}$ for six equally spaced points in $[0,1]$.  To graph this data, the size of the graph must first be decided.  Choosing a width of $144\mathrm{\ bp}$ and a height of $89\mathrm{\ bp}$, a minimally controlled plot (as in Figure \ref{fig:data}) of this data can be generated by
\begin{lstlisting}[xleftmargin=38bp]
draw begingraph(144bp,89bp);
   gdraw "data.d";
endgraph;
\end{lstlisting}
The \texttt{graph} package provides many commands used to customize generated graphs, and these commands are fully documented in the manual \cite{hobby:graph} for the \texttt{graph} package.
\begin{figure}[hptb]
   \begin{center}\textattachfile[color={0 0 0},mimetype={text/plain}]{data.mp}{\includegraphics{data.mps}}\end{center}
   \caption{$f(x)=\sqrt{x}$ using the \texttt{graph} package}\label{fig:data}
\end{figure}
\end{section}
